\documentclass[a4paper,12pt]{scrartcl}
\usepackage[utf8]{inputenc}
\usepackage[ngerman]{babel}
\usepackage[T1]{fontenc}
\usepackage{amsmath}
\usepackage{stmaryrd}
\usepackage{wasysym}
\usepackage{lmodern}
\usepackage{graphicx}
\usepackage{paralist}
\usepackage{upgreek}
\usepackage{subfigure}
\usepackage{tipa}
\usepackage{amssymb}
\usepackage{gensymb}
\usepackage{dsfont}
\usepackage{mathtools}
\usepackage{ stmaryrd }
\usepackage{fancyhdr}

%\title{Abgabe 1}
%\author{Rafael Heid, Julian Deinert, Sabrina Buczko Gruppe\\ 6 und 7}
%\date{Abgabe am 24.10.16}

\gdef\blatt{FGI-2 Aufgabenblatt 04}

\title{\blatt}
\date{Gruppe 06}
\author{Sabrina Buczko 6663234, Julian Deinert 6535880, Rafael Heid 6704828}


\pagestyle{fancy}
\fancyhf{}
\fancyhead[L]{\blatt}
\fancyhead[R]{Buczko, Deinert, Heid}
\fancyfoot[C]{\thepage}

\begin{document}
\maketitle
\newpage
\setcounter{section}{3}
\section{}
\setcounter{subsection}{2}
\subsection{}
\subsubsection{}
$L(TS_{Enterprise})= ((du)^*+(le)^*+((lk)\cdot (p)^*\cdot (bwe))^*)^*$\\
$L^\omega(TS_{Enterprise})=((du)^*+(le)^*+((lk)\cdot (p)^*\cdot (bwe))^*)^\omega$
\subsubsection{}
%$SS(M_{Enterprise})=(s_0(s_1 + s_2 + (s_2((s_3)^+ s_4))^*))^\omega$\\
$SS(M_{Enterprise})=(0(1 + 2 + (2((3)^+ 4))^*))^\omega$\\

\subsubsection{}
$E_S(s_0)=\{orbit\}$\\
$E_S(s_1)=\{away, orbit\}$\\
$E_S(s_2)=\{warp\}$\\
$E_S(s_3)=\{shields\}$\\
$E_S(s_4)=\{\emptyset\}$\\
$E_S(SS(M_{Enterprise}))=E_S((0(1 + 2 + (2((3)^+ 4))^*))^\omega)$\\
$= ((E_S(s_0)(E_S(s_1) + E_S(s_2) + (E_S(s_2)((E_S(s_3))^+ E_S(s_4)))^*))^\omega)$\\
$= (\{orbit\}(\{away,orbit\} + \{warp\} + (\{warp\}((\{shields\}^+ \emptyset))^*))^\omega$\\
\subsubsection{}

\subsubsection{}
\subsection{}
%1. irgendwann einmal gilt immer warp oder shields
%2.
%3. shields impliziert immer shields until das nächtse mal warp kommt
%4. orbit impliziert immer dass das übernächste mal warp kommt
%5 wenn nicht orbit,nicht shields, nicht warp und das übernchste mal nicht orbit kommt impliziert dies immer dass iwann mal shields kommt
%6. beim überübernächsten Mal gilt irgendwann orbit und nicht warp

\begin{tabular}{|c|>{\(}c<{\)}|c|c|}
\hline
\# & f & $M_{Enterprise} \models f$ & $M_{Enterprise},\pi \models f$ \\
\hline
1 & $\diamond \square(warp\vee shields)$ & Falsch & Richtig \\
2 & $\square\diamond(warp\vee shields)$ & Falsch & Richtig \\
3 & $\square(shields\Rightarrow(shields \textbf{U}(\circ warp)))$ & Falsch & Falsch \\
4 & $\square(orbit\Rightarrow\circ\circ warp)$ & Falsch & Falsch\\
5 & $\square((\neg orbit\wedge\neg shields\wedge\neg warp\wedge\circ\circ\neg orbit)\Rightarrow\diamond shields)$ & Richtig & Richtig \\
6 & $\circ\circ\circ\diamond(orbit\wedge\neg warp)$ & Falsch & Falsch\\ 
\hline
\end{tabular}
\textbf{Begründungen:}\\
1. $M\nvDash f$, da in einem Pfad $(s_0 s_1)^\omega$ niemals warp oder shields gilt.\\
$M,\pi\vDash f$, da $s_2$ und $s_3$ in $\pi$ unendlich oft durchlaufen werden und sie die Etiketten warp und shields haben.\\\\
2. $M\nvDash f$, aus dem gleichen Grund wie in 1.\\
$M,\pi\vDash f$, aus dem gleichen Grund wie in 1.\\\\
3. $M\nvDash f$, denn wenn shields aktiviert ist in $s_3$ und das nächste mal warp in $s_2$ kommt, ist shields schon vorher wieder in $s_4$ negiert worden.\\
$M,\pi\nvDash f$, aus dem selben Grund wie in $M\nvDash f$.\\\\
4. $M\nvDash f$, da $M\vDash f$ nur gilt in dem Pfad $(s_0 s_1 s_0 s_2)^\omega$. In einem Pfad $(s_0 s_1 s_0 s_1)^\omega$ impliziert orbit niemals dass das übernächste Mal warp kommt.\\
$M,\pi\nvDash f$, da man nach dem $s_0$ als übernächstes Mal im $s_3$ ist, ist warp negiert.\\\\
5. $M\vDash f$, da orbit, shields und warp nur in Zustand $s_4$ negiert sind und beim übernächsten Mal orbit immernoch negiert sein soll, können wir uns nur in $s_3$ befinden. Und in $s_4$ gilt shields.\\
$M,\pi\vDash f$, da im Pfad $\pi$, nach dem Zustand $s_4$ immer als übernächster Zustand $s_3$ kommt gilt die selbe Begründung wie bei $M\vDash f$.\\\\
6. $M\nvDash f$, da in einem Pfad der höchstens zweimal den Zustand $s_0$ durchläuft, nie mehr orbit und nicht warp gilt.\\
$M,\pi\nvDash f$, da $s_0$ nur am Anfang durchlaufen wird und dann der Pfad $\pi$ in der Schleife bleibt, gilt nie wieder orbit und nicht warp.
\end{document}
