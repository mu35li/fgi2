\documentclass[a4paper,12pt]{scrartcl}
\usepackage[utf8]{inputenc}
\usepackage[ngerman]{babel}
\usepackage[T1]{fontenc}
\usepackage{amsmath}
\usepackage{stmaryrd}
\usepackage{wasysym}
\usepackage{lmodern}
\usepackage{graphicx}
\usepackage{paralist}
\usepackage{upgreek}
\usepackage{subfigure}
\usepackage{tipa}
\usepackage{amssymb}
\usepackage{gensymb}
\usepackage{dsfont}
\usepackage{mathtools}
\usepackage{ stmaryrd }
\usepackage{fancyhdr}

%\title{Abgabe 1}
%\author{Rafael Heid, Julian Deinert, Sabrina Buczko Gruppe\\ 6 und 7}
%\date{Abgabe am 24.10.16}

\gdef\blatt{FGI-2 Aufgabenblatt 04}

\title{\blatt}
\date{Gruppe 06}
\author{Sabrina Buczko 6663234, Julian Deinert 6535880, Rafael Heid 6704828}


\pagestyle{fancy}
\fancyhf{}
\fancyhead[L]{\blatt}
\fancyhead[R]{Buczko, Deinert, Heid}
\fancyfoot[C]{\thepage}

\begin{document}
\maketitle
\newpage
\setcounter{section}{3}
\section{}
\setcounter{subsection}{2}
\subsection{}
\subsubsection{}
$L(TS_{Enterprise})= ((du)^*+(le)^*+((lk)\cdot (p)^*\cdot (bwe))^*)^*$\\
$L^\omega(TS_{Enterprise})=((du)^*+(le)^*+((lk)\cdot (p)^*\cdot (bwe))^*)^\omega$
\subsubsection{}
%$SS(M_{Enterprise})=(s_0(s_1 + s_2 + (s_2((s_3)^+ s_4))^*))^\omega$\\
$SS(M_{Enterprise})=(0(1 + 2 + (2((3)^+ 4))^*))^\omega$\\

\subsubsection{}
$E_S(s_0)=\{orbit\}$\\
$E_S(s_1)=\{away, orbit\}$\\
$E_S(s_2)=\{warp\}$\\
$E_S(s_3)=\{shields\}$\\
$E_S(s_4)=\{\emptyset\}$\\
$E_S(SS(M_{Enterprise}))=E_S((0(1 + 2 + (2((3)^+ 4))^*))^\omega)$\\
$= ((E_S(s_0)(E_S(s_1) + E_S(s_2) + (E_S(s_2)((E_S(s_3))^+ E_S(s_4)))^*))^\omega)$\\
$= (\{orbit\}(\{away,orbit\} + \{warp\} + (\{warp\}((\{shields\}^+ \emptyset))^*))^\omega$\\
\subsubsection{}
Da \textit{shields} nur in $s_3$ gilt, ist $Sat(shields)=\{3\}$.\\
$\neg orbit$ gilt in $s_2, s_3$ und $s_4$ 
also ist $Sat(\neg{orbit})=\{2,3,4\}$.\\
Schließlich gilt $warp$ nur in $s_2$ und somit ist $Sat(warp)=\{2\}$.\\

In natürlicher Sprache bedeutet
$f=$ G $(shields \Rightarrow (($ X $ \neg{shields} \Rightarrow $ F $warp))$,
 dass immer wenn shields gilt, und als nächstes shields nichtmehr gilt, 
irgendwann in der Zukunft warp gilt. Anders gesagt: Damit warp genutzt 
werden kann, müssen die shields vorher deaktiviert werden, sofern sie 
aktiviert sind.
\subsubsection{}
\subsection{}
1. irgendwann einmal gilt immer warp oder shields
2.
3. es gilt immer, dass shields impliziert shields until das nächtse mal warp kommt
4. orbit impliziert immer dass das übernächste mal warp kommt
5 wenn nicht orbit,nicht shields, nicht warp und das übernchste mal nicht orbit kommt impliziert dies immer dass iwann mal shields kommt
6. beim überübernächsten Mal gilt irgendwann orbit und nicht warp
%
%\begin{tabular}{|c|>{\(}c<{\)}|c|c|}
%\hline
%\# & f & \(\Mcell \models f\) & \(\Mcell,\pi \models f\) \\
%\hline
%1 & \tlG(\eActive)
%& Falsch & Falsch \\
%2 & \tlG\tlF(\eActive)
%& Falsch & Wahr \\
%3 & \tlG(\tlX \eActive \Rightarrow \eOn)
%& Wahr & Wahr \\
%4 & \tlG\tlF(\eActive \Rightarrow \tlX\tlX \lnot \eOn)
%& Wahr & Wahr \\
%5 & \tlG\tlF(\lnot \eBattery \lor \eActive \lor \lnot \eOn \lor \eError)
%& Falsch & Wahr \\
%6 & \tlX\tlX\tlX \eActive
%& Wahr & Wahr \\
%\hline
%\end{tabular}

\end{document}
