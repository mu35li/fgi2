\documentclass[a4paper,12pt]{scrartcl}
\usepackage[utf8]{inputenc}
\usepackage[ngerman]{babel}
\usepackage[T1]{fontenc}
\usepackage{amsmath}
\usepackage{stmaryrd}
\usepackage{wasysym}
\usepackage{lmodern}
\usepackage{graphicx}
\usepackage{paralist}
\usepackage{upgreek}
\usepackage{subfigure}
\usepackage{tipa}
\usepackage{amssymb}
\usepackage{gensymb}
\usepackage{dsfont}
\usepackage{mathtools}
\usepackage{ stmaryrd }
\usepackage{fancyhdr}
\usepackage{tikz}
\usetikzlibrary{arrows,automata}

%\title{Abgabe 1}
%\author{Rafael Heid, Julian Deinert, Sabrina Buczko Gruppe\\ 6 und 7}
%\date{Abgabe am 24.10.16}

\gdef\blatt{FGI-2 Aufgabenblatt 08}

\title{\blatt}
\date{Gruppe 06}
\author{Sabrina Buczko 6663234, Julian Deinert 6535880, Rafael Heid 6704828}


\pagestyle{fancy}
\fancyhf{}
\fancyhead[L]{\blatt}
\fancyhead[R]{Buczko, Deinert, Heid}
\fancyfoot[C]{\thepage}

\begin{document}
\maketitle
\newpage
\setcounter{section}{7}
% Section 7
\section{}
\setcounter{subsection}{2}
% Section 7.3
\subsection{}
\subsubsection{}
\begin{tikzpicture}[->,>=stealth',shorten >=1pt,auto,node distance=3cm, semithick]
  \tikzstyle{every state}=[fill=none,draw=none,text=black]

  \node[state] (A)              {$p1(2)$};
  \node[state] (B) [right of=A] {$p1p2$};
  \node[state] (C) [below of=B] {$p2p3$};
  \node[state] (D) [right of=B] {$p2(2)$};
  \node[state] (E) [below of=D] {$p2p5$};
  \node[state] (F) [below of=A] {$p1p3$};
  \node[state] (G) [below of=E] {$p4p5$};
  \node[state] (H) [below of=C] {$p3p4$};
  \node[state] (I) [below of=F] {$p3(2)$};
  \node[state] (J) [below of=H] {$p3p6$};
  \node[state] (K) [below of=G] {$p5p6$};
  \node[state] (L) [right of=E] {$p2p6$};
  \node[state] (M) [below of=L] {$p4p6$};
  \node[state] (N) [right of=D] {$p2p4$};
  \node[state] (O) [below of=K] {$p5(2)$};
  \node[state] (P) [below of=J] {$p3p5$};
  \node[state] (Q) [below of=M] {$p6(2)$};
  \node[state] (R) [right of=M] {$p4(2)$};
     

  \path (A) edge              node {$t_1$} (B)
  		(A) edge			  node {$t_2$} (F)
        (B) edge              node {$t_1$} (D)
            edge              node {$t_2$} (C)
        (C) edge              node {$t_3$} (A)
        (E) edge 			  node {$t_6$} (C)
        (F) edge              node {$t_1$} (C)
        	edge              node {$t_1$} (I)
        (C) edge              node {$t_4$} (G)
        (G) edge              node {$t_5$} (E)
        	edge              node {$t_6$} (H)
        	edge              node {$t_7$} (Q)
        (H) edge              node {$t_5$} (C)
        (J) edge              node {$t_8$} (H)
        	edge              node {$t_9$} (P)
        (K) edge              node {$t_6$} (J)
        	edge              node {$t_8$} (G)
        	edge              node {$t_9$} (O)
        (L) edge              node {$t_9$} (E)
        	edge              node {$t_8$} (N)
        (M) edge              node {$t_9$} (G)
        (N) edge              node {$t_5$} (D)
        (O) edge              node {$t_6$} (P)
        (P) edge              node {$t_6$} (I)
        (Q) edge              node {$t_8$} (M)
        	edge              node {$t_9$} (K)
        (M) edge              node {$t_8$} (R);
  
        
\end{tikzpicture}
%Anhand des Erreichbarkeitsgraphen ist zu erkennen, dass das Netz \textbf{nicht reversibel} ist, da $m_0$ in $p_1$ von allen weiteren erreichbaren Markierungen aus erreichbar ist. Ebenso ist es \textbf{lebendig}, da jede Transition in jeder Markierung potenziell aktivierbar ist und es keine Verklemmungen gibt. Es ist auch \textbf{beschränkt}, da der Graph endlich ist.\\\\%
\\
\textbf{Lebendigkeit:} \\
\textbf{Reversibilität:} \\
\textbf{Beschränktheit:} \\
\textbf{Strukturelle Eigenschaften:\\}
\subsubsection{}
\textbf{Strukturelle Beschränktheit:}\\
\textbf{Strukturelle Lebendigkeit:}\\
\textbf{Fairness: }Das Netz schaltet nicht fair, da z.b. von Markierung $p_2 p_3$ aus die beiden Marken nur reichen um entweder den Pfad $t_3$ zu gehen oder den Pfad $t_4$. Es gibt also unfaire Folgen in diesem Netz. Also müssen nicht alle Transitionen immer schalten, was für Fairness eine Voraussetzung ist.
\subsubsection{}
ee
\subsubsection{}
eee
\subsection{}
\textbf{1. Erreichbarkeitsgraph zeichnen}\\
\begin{tikzpicture}[->,>=stealth',shorten >=1pt,auto,node distance=3cm, semithick]
  \tikzstyle{every state}=[fill=none,draw=none,text=black]

  \node[state] (A)              {$p1(2)$};
  \node[state] (B) [right of=A] {$p1p2(3)$};
  \node[state] (C) [below of=A] {$p1p2(2)p3$};
  \node[state] (D) [right of=B] {$p2(6)$};
  \node[state] (E) [below of=D] {$p2(5)p3$};
  \node[state] (F) [below of=B] {$p2(4)p3(2)$};
  \node[state] (G) [below of=C] {$p1p2(2)$};
  \node[state] (H) [below of=G] {$p2(5)$};
  \node[state] (I) [right of=H] {$p2(4)p3$};
  \node[state] (J) [below of=F] {$p2(3)p3(3)$};
  \node[state] (K) [below of=E] {$p2(2)p3(4)$};
  \node[state] (L) [below of=K] {$p1p2p3(2)$};
  \node[state] (M) [right of=K] {$p1p3(3)$};
  \node[state] (N) [below of=I] {$p2(3)p3(2)$};
  \node[state] (O) [below of=L] {$p1p2p3$};
  \node[state] (P) [below of=O] {$p1p2$};
  \node[state] (Q) [right of=P] {$p2(4)$};
  \node[state] (R) [right of=O] {$p2(3)p3$};
  \node[state] (S) [right of=L] {$p2(2)p3(2)$};
  \node[state] (T) [right of=M] {$p1p3(2)$};
  \node[state] (U) [right of=S] {$p1p3$};
  
  \node[state] (W) [below of=U] {$p1$};
  \node[state] (X) [right of=W] {$p2(3)$};
  \node[state] (Z) [below of=W] {$p2(2)p3$};
\node[state] (V) [below of=Z] {$p2(2)p3(3)$};
   
  

  \path (A) edge              node {$c$} (B)
        (B) edge              node {$d$} (C)
            edge              node {$c$} (D)
        (C) edge              node {$b$} (A)
        	edge              node {$a$} (G)
        (D) edge              node {$d$} (E)
        (E) edge              node {$b$} (B)
        	edge              node {$d$} (F)
        (F) edge              node {$b$} (C)
        	edge              node {$d$} (J)
        (G) edge              node {$c$} (H)
        (H) edge              node {$d$} (I)
        (I) edge              node {$b$} (G)
        	edge              node {$d$} (N)
        (J) edge              node {$d$} (K)
        	edge              node {$b$} (L)
        (K) edge              node {$b$} (M)
        (L) edge              node {$a$} (O)
        (M) edge              node {$a$} (T)
        (N) edge              node {$b$} (O)
        	edge          [bend right]    node {$d$} (V)
        (O) edge              node {$a$} (P)
        (P) edge              node {$c$} (Q)
        (Q) edge              node {$d$} (R)
        (R) edge              node {$d$} (S)
        	edge  node {$b$} (P)
        (S) edge              node {$b$} (U)
        (T) edge              node {$a$} (U)
        (U) edge              node {$a$} (W)
        (W) edge              node {$c$} (X)
        (X) edge              node {$d$} (Z)
        (Z) edge              node {$b$} (W)
        (V) edge      [bend right=100]        node {$b$} (T);
\end{tikzpicture}
\textbf{2. SZKs finden}\\
\textbf{3. terminale SZKs finden}\\
Zu den terminalen SZKs gehört nur der Zyklus mit den Markierungen $p_1$, $p_2 (3)$, und $p_2 (2)p_3$.\\
\textbf{4. Prüfe für jede terminale SZK, ob es ein m' enthält, dass das Prädikat erfüllt.} \\
In der einzigen terminalen SZK kann man von den drei Markierungen $p_1$, $p_2 (3)$,$p_2 (2)p_3$ in kein m' gelangen in der Transition a aktivierbar ist. Also ist a nicht lebendig. Die Transitionen b,c und d hingegen sind von jeder m erreichbaren m' aktivierbar und somit auch in unserer terminalen SZK aktivierbar. Also sind b,c und d lebendig.
\subsection{}
\subsection{}
Angenommen N besitzt die Siphon/Trap-Eigenschaft, dann aktiviert jede erreichbare Markierung...?
\begin{itemize}
\item{mind. eine Transition}
\item{höchstens eine Transition}
\item{keine Transition}
\item{beliebig viele Transitionen}
\end{itemize}\\
Wenn Petrinetze so mächtig wie Turing-Maschinen wären, hätte dies den Nachteil, dass Beschränktheit, Erreichbarkeit und Lebendigkeit nicht entscheidbar sind.
\begin{itemize}
\item{wahr}
\item{falsch}
\end{itemize}\\
Sind gefärbte Netze mit endlichen Farbmengen turing-mächtig?
\begin{itemize}
\item{ja}
\item{nein}
\end{itemize}\\
Sind gefärbte Netze mit beliebigen Farbmengen turing-mächtig?
\begin{itemize}
\item{ja}
\item{nein}
\end{itemize}\\
Der Vektor $\Delta_{\mathcal{N}}(t)\in \mathds{Z}^{|P|}$ der Transition $t\in T$ heißt...?
\begin{itemize}
\item{Ordnung}
\item{Wirkung}
\item{Index}
\item{Lösung}
\item{Invariante}
\end{itemize}\\
\end{document}