\documentclass[a4paper,12pt]{scrartcl}
\usepackage[utf8]{inputenc}
\usepackage[ngerman]{babel}
\usepackage[T1]{fontenc}
\usepackage{amsmath}
\usepackage{stmaryrd}
\usepackage{wasysym}
\usepackage{lmodern}
\usepackage{graphicx}
\usepackage{paralist}
\usepackage{upgreek}
\usepackage{subfigure}
\usepackage{tipa}
\usepackage{amssymb}
\usepackage{gensymb}
\usepackage{dsfont}
\usepackage{fancyhdr}

%\title{Abgabe 1}
%\author{Rafael Heid, Julian Deinert, Sabrina Buczko Gruppe\\ 6 und 7}
%\date{Abgabe am 24.10.16}

\gdef\blatt{FGI-2 Aufgabenblatt 02}

\title{\blatt}
\date{Gruppe 06}
\author{Sabrina Buczko 6663234, Julian Deinert 6535880, Rafael Heid 6704828}


\pagestyle{fancy}
\fancyhf{}
\fancyhead[L]{\blatt}
\fancyhead[R]{Buczko, Deinert, Heid}
\fancyfoot[C]{\thepage}

\begin{document}
\maketitle
\newpage
\setcounter{section}{1}
\section{}
\setcounter{subsection}{2}
\subsection{}
\subsubsection{}
$L(A_{2.3})=\lambda+(ab)^*+cd^*$\\
$L^\omega(A_{2.3}) = (ab)^\omega + cd^\omega$\\
$(L(A_{2.3})=\lambda+(ab)^*+cd^*)^\omega$
\subsubsection{}
$L^\omega(A_{2.3})$ ist die akzpetierte Sprache, wenn wir den NFA $A_{2.3}$ als Büchi-Automaten betrachten bei dem mindestens ein Endzustand unendlich oft durchlaufen werden muss. Der Automat akzpetiert die $\omega$-Wörter $\omega_1 = (ab)^\omega$ und $\omega_2 = cd^\omega$. $(L(A_{2.3}))^\omega$ ist eine Sprache mit unendlich vielen $\omega$-Wörtern. Diese können beliebig aus Teilen der vom NFA $A_{2.3}$ akzpetierten Sprache $L(A_{2.3})$ zusammengesetzt werden. Beispiele für solche Wörter sind $\omega_1 = (ab)^\omega$ oder auch $\omega_3 = (abc)^\omega$.
\end{document}
